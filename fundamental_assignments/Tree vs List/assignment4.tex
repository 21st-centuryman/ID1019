\documentclass[a4paper,11pt]{article}

\usepackage[utf8]{inputenc}

\usepackage{minted}

\begin{document}

\title{
    \textbf{Taking the derivative}
}
\author{Alexander Berg}
\date{Spring Term 2022}

\maketitle

\section*{Introduction}

For the fourth lab exercise of this course the author was told to compare and contrast the difference between creating an ordered tree and a ordered list, in order to achieve this the author will utilized a benchmark function which was provided to him by the lecturer. 

\section*{Tree vs list}

In order to properly explain this assignment, the author will firstly explain the main differences between these functions, starting with a list:

\begin{minted}{elixir}
  def list_insert(e, []) do [e] end
  def list_insert(e, [h|t]) do
    case e do
      e when e <= h -> [e,h|t]
      _ -> [h|list_insert(e, t)]
    end
  end
\end{minted}

Note that this is not the only function for lists as {\tt list\_new} initializes the list, but only returns an empty list. 
These functions are the {\tt list\_insert} used to organize a value added to our lists. The first function is used when the list is empty, returning the value of {\tt e} in the list. The second function is used when there are values in the list. Since we want the list to be ordered we need to compare e to {\tt h} in an case statement to see if we should add the value of {\tt e} to the head of the list. If the first case is returned we add {\tt e} to the head of the list. Otherwise we recursively call the function and use the next value {\tt t} in the list, we continue this process going head to tail until we find the right spot for the value of {\tt e}. This process can become quite tedious which is why we will compare it with a tail structure later in this document.

\begin{minted}{elixir}
  def tree_insert(e, :nil) do{:leaf, e}end
  def tree_insert(e, {:leaf, head}=right) do if e < head do
    {:node, e, :nil, right} end end
  def tree_insert(e, {:leaf, _}=left) do
    {:node, e, left, :nil} end

  def tree_insert(e, {:node, head, left, right}) do
    case e do
      e when e < head -> {:node, head, tree_insert(e, left), right}
      _ ->{:node, head, left, tree_insert(e, right)}
    end
  end


\end{minted}

Note that the function {\tt tail\_new} is missing as it does not concern this comparison.
These functions can effectively be split into 2 parts, first are the first three which concern empty trees or if there is a single head (node). It will then insert the new node as the head of the tree and make the current head a "leaf", if the node is smaller than the current head the same process will ofccur with the second function. In that case
The last function concerns all other cases and is structured similarly to our list function earlier, calling {\tt tree\_insert} recursivly comparing our new node with the rest of the nodes, starting with the root until the right place is found for the node.

\section*{Benchmarks}

With both a tree structure and lists explained, it is time to use the included benchmark to check the execution time of these structures, the result of which is shown below:

\begin{table}[h]
  \centering
  \renewcommand{\arraystretch}{1.3}
  \begin{tabular}{ p{2cm} l l  p{5cm} } 
      \multirow{2}{10cm}{\textbf{N}} & \multicolumn{2}{ c }{\textbf{Function [\textit{µs}]}} \\ 
      \cline{2-3}
      & {\tt list\_insert}    & {\tt tree\_insert}  \\    
      \hline\hline
      $2^4$    & 235          & 145      \\
      $2^5$    & 340          & 191      \\
      $2^6$    & 1223          & 454     \\
      $2^7$    & 4247         & 778     \\
      $2^8$    & 15060        & 1974     \\
      $2^9$    & 61163        & 6809    \\
      $2^{10}$   & 248668       & 25671    \\
      $2^{11}$   & 1584427       & 118358    \\
      $2^{12}$   & 5824127      & 210739   \\
      $2^{13}$   & 24057775     & 475440   \\ [0.5ex]
    \hline
  \end{tabular}
  \label{t1}
\end{table}

As can be seen in the result the differences at first are quite slim, but as time increases the differences grows. This occurs since a list is a collection of nodes where every node only refers to one neighbor. A tree structure on the other hand has the ability to refer to mulitple neighbor should that be nesseary. However it should be noted that during these test runs the numbers are randomized and it is not a given that the differences should be this vast. However we can conclude that for long lists of data a tree structure would be superior to that of lists.
This is not to say that lists are inferior to that of trees. Looking at the code it requires much fewer lines to write a list than that of a tree, making it suitable for small data where readability is preferred over execution time.  

\section*{Conclusion}

This assignment helped explain to the author the main differences between lists and tree data structures. Previously the author understood how they are different structurally but conceptually the differences with tests such as execution time helped drive how the reason for when and how to use both data structures in an effective matter.

\end{document}
